%!TEX root = /Users/oroce/corvinus/Anonimity-TDK/dolgozat.tex

Az internet penetráció növekedése (TODO hivatkozás) és az információs társadalom fejlődése miatt a felhasználók egyre több szolgáltatásból választják az online verziót, egyre több időt töltenek (TODO hivatkozás erről?) az internetre csatlakozva.\\
Az online szolgáltatások széles választéka, és az azokat használó felhasználók adatainak eltulajdonításáról (banki adatok, számlaszámok) rengeteg cikk, tudományos munka készült már, ez a dolgozat azt szeretné bemutatni, hogy az interneten tevékenykedve milyen, szinte láthatatlan nyomokat hagynak maguk után a felhasználók, ezeket kik és hogyan használják ki.\\
A dolgozat bemutatja, mind felhasználói, mind szolgáltatói szemszögből mire kell odafigyelni (a felhasználóknak mit érdemes elrejteni, a szolgáltatóknak mit érdemes monitorozni), hogy a lehető legkevesebb vagy éppen a legtöbb információhoz cseréljen gazdát.
\\
A dolgozat végén pedig, egy, a felhasználói profil összeállítására alkalmas szoftver tervezését és megvalósítását ismerheti meg az olvasó.
\clearpage

\section{Anonimitás felhasználói oldalról} % (fold)
\label{sec:anonimitás_felhasználói_oldalról}

Mivel az internet Magyarországon a rendszerváltás után jelent (TODO első szolgáltató), ezért a felhasználók tudása, és oktatása nem fejlődött az internet sebességével. Ezért sajnos az internetet böngésző felhasználók gyakran nincsenek tisztában, hogy milyen sok mindent elárulnak magukról egy-egy kattintással, elfogadnak olyan kéréseket, amelyeket el sem olvasnak, illetve megbíznak a weboldalakban.\\
Természetesen léteznek weboldalak, amelyek még a gyakorlott, az anonimitással teljesen tisztában lévő haladó felhasználókat is csapdába csalják.\\
A dolgozat megpróbál rámutatni, azokra biztonsági szempontokra, amelyeket szem előtt tartva a felhasználót sokkal kevesebbet fog elárulni magáról a böngészése során. Többek között a következő témákat érintve: privát böngészés, külső sütik, HTML5-, RIA veszélyei és lehetőségei.

\section{Anonimitás üzleti oldalról} % (fold)
\label{sec:anonimitás_üzleti_oldalról}

Az üzleti oldal természetesen teljesen más oldalról közelít az anonimitáshoz, egy weboldalnak tudnia kell monitorozni a felhasználóit, egy hírportál esetén releváns, célzott reklámokat kell tudni megjeleníteni, amelyhez szükséges egy minél pontosabb felhasználói profil felállítása.\\
A dolgozatban bemutatásra kerülnek azok a technikák, technológia lehetőségek, melyekkel a felhasználók minél könnyebben, pontosabban beazonosíthatók. Továbbá megvizsgálásra kerül a Facebook, melynek segítségével a felhasználói profil pontosítható.
% section anonimitás_üzleti_oldalról (end)
\section{A téma létjogosultsága, megéri vele foglalkozni?} % (fold)
\label{sec:a_téma_létjogosultsága}
Az anonimitás olyan webes alkalmazásoknál, ahol van regisztráció - és kötelező is regisztrálni (email alkalmazások, közösségi média) - természetesen nem kap komoly hangsúlyt, ugyanis tisztában vannak a felhasználóik adataival.\\
Azonban online hírportáloknál, keresőmotoroknál, ahol a tartalom ingyenesen elérhető és az elsődleges bevétel a reklámokból van, ott kimondottan fontosat szerepet kap a felhasználói profilok felépítése.\\
Az analitika rendkívül fontos ilyen weboldalak esetén, azonban nem képes arra, hogy megmondja az oldalra látogató felhasználókból, hogy volt-e már az oldalon, illetve mennyit töltött és milyen tartalmak érdeklik.


% section a_téma_létjogosultsága (end)
% section anonimitás_felhasználói_oldalról_monitorozás_üzleti_oldalról_megéri_vele_foglalkozni_ (end)