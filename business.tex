%!TEX root = /Users/oroce/corvinus/Anonimity-TDK/dolgozat.tex
\section{A szereplők} % (fold)
\label{sec:a_szereplők}

Az üzleti szempontoknál érdemes megvizsgálni, hogy egyáltalán kik azok a szereplők a piacon akik szeretnék a felhasználókat azonosítani. Az elsődleges célja a felhasználók azonosításának, hogy az érdeklődésnek megfelelő tartalmat tudjon nyújtani a szolgáltató, ennek a legfontosabb, és üzletileg a legjelentősebb formája a reklámok.\hfill\\

Persze felmerül a kérdés, hogy a felhasználók hol találkozhatnak reklámokkal a weben, a válasz tulajdonképpen az, hogy mindenhol, a dolgozat azonban következő - jelenleg a legfontosabb - területekre koncentrál: közösségi oldalak, hír- és hírgyűjtő oldalak illetve keresők.

\subsubsection{Közösségi oldalak} % (fold)
\label{ssub:közösségi_oldalak}

A közösségi oldalak a reklámok szempontjából egy speciális területnek számítanak, ugyanis a felhasználók nem anonim módon (értsd bejelentkezés nélkül) böngésznek az oldalon, hanem egy bejelentkezés után. Továbbá az közösségi oldalakon a felhasználók már regisztrációkor megosztják az alapvető információkat magukról (név, születési dátum, lakhely, foglalkozás), illetve az oldalon történő tevékenységeikkel további adatokat adnak ki magukról. Összességében elmondható, hogy az oldalak itt nem kell az anonim felhasználókból megépíteni a felhasználói profilokat, hiszen a tagok önként megteszik, amelynek megbízhatósága, pontossága sokkal jobb minőségű, mint az anonim azonosítás.\hfill\\

A Facebook, mint a legnagyobb közösségi oldal (2011. decemberi adatok közel 845 millió havi aktív felhasználói aktivitást jeleztek (\cite{fb_stat})), nagyon szigorú specifikációban köti ki a megjeleníthető reklámok kinézetét, formátumát, szövegezését, cserébe viszont a nagyszámú profilnak köszönhetően a hirdetéseket lehető legpontosabban képes célozni (targetálni). \cite{thinkdigital_fb}
% subsubsection közösségi_oldalak (end)

\subsubsection{Hír- és hírgyűjtő oldalak} % (fold)
\label{ssub:hír_és_hírgyűjtő_oldalak}
A pusztán internetes híroldalaknál az legfontosabb bevétel a reklámokból származik, ezért rendkívül fontos szerepet kap a felhasználói profilok felállítása, és a célzott reklámok megjelenítése. A profilok felállítására létezik sok kutatóintézet és internetes piackutató cég, akik kész megoldásokat kínálnak a portáloknak.\hfill\\
A Közép-Európai régióban a legnagyobb ilyen jellegű felméréseket, monitorozást végző szervezete a Gemius (\url{http://gemius.hu}), illetve fontos kiemelni, a jelenleg egyre több elismerést elnyerő magyarországi feltörekvő céget, a Playertise szolgáltatás (\url{http://playertise.com}), amely a hírportálon (és egyéb, akár személyesen oldalakon) megjelenő videók tartalma, és a felhasználói profil képes a hirdetéseket megjeleníteni.\hfill\\
\\
Fontos kiemelni, hogy a közösségi oldalak egyre több problémát okoznak a hírportáloknak, mivel a sokkal pontosabb felhasználó profilok, és nagyobb számú célzott reklámok száma a hirdetőket elcsábítják. \cite{newspaper_ad_revenue}\hfill\\
Ennek köszönhetően egyre több hírportál kezdi bevezetni - a reklámok mellett - a fizetési modelljét, amellyel próbálják pótolni a hirdetésekből csökkenéséből származó veszteségeiket. \cite{newspapers_go_to_paying_model}
% subsubsection hír_és_hírgyűjtő_oldalak (end)

\subsubsection{Keresők} % (fold)
\label{ssub:keresők}
A legnagyobb webes kereső oldalak rendelkezek saját megoldással reklámok, és a hirdetések kiszolgálására (Google - AdWords, Yahoo - Advertising, Bind - AdCenter). Természetesen a közösségi oldalak által egyre nagyobb szelet a keresőket is megviseli, ezért próbálnak a keresők is minél pontosabb profilokat összeállítani, illetve a lehető információt kinyerni a felhasználók böngészőiből, akár a megoldásaikkal már a szürke zónában ténykedve. \cite{google_on_safari}\hfill\\
\\
Mivel lehet még tökéletesebb profil összeállítani?
\\
Természetesen a regisztrációval. A Google rengeteg egyéb szolgáltatással rendelkezik a keresőjén kívül (email, blog olvasó, mobil- és asztali operációs rendszer, fordító, stb.), amelyekkel megpróbálja rábírni a felhasználót a regisztrálásra, majd a kumulált adatokból már egy olyan profilhoz jut, amelynek segítségével a hirdetések átkattintási aránya (clickthrough rate) magasabb lesz. Továbbá a Google a regisztrált felhasználók részére, olyan plusz szolgáltatások érhetőek el, mint a személyre szabottabb keresési lehetőség, amely a Google+ közösségi oldalon folytatott tevékenység alapján rendezi sorrendbe a találatokat. \cite{gplus_personal}
% subsubsection keresők (end)
% section a_szereplők (end)