%!TEX root = /Users/oroce/corvinus/Anonimity-TDK/dolgozat.tex

Összesítve elmondható, hogy webes anonimitás talán minden korábbi szintjénél komolyabb szerepet kaphat az egyre bővülő online szolgáltatások széles palettája és lehetőségei miatt. De ami talán a legfontosabb, hogy a felhasználóknak meg kell tanulniuk, hogy milyen módszerekkel tudják magukat nagyobb biztonságba helyezni, ugyanis olyan cégek sem riadnak vissza a szürke zónától, mint a Google, ha van lehetőségük a felhasználóról információkat gyűjteni. Továbbá a törvényhozók is elkövetnek mindent, hogy felhasználók monitorozása a hatóságok szeme elől sem lehessen rejtve.\hfill\\
Továbbá fontos kiemelni a közösségi oldalakat, amelyekre a felhasználók önkéntesen feltöltik az adataikat, nem törődve azzal, hogy az oldalak üzemeltetői továbbadják-e harmadik félnek, illetve mit tesznek az adataikkal. Sajnos az internetes-biztonság tudás és oktatás alanya szintje miatt, a weboldalak könnyen kihasználhatják a felhasználók jóhiszeműségét, amellyel fontos információkhoz juthatnak a profilok felépítését illetően.
